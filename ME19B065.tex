\documentclass{article}
\usepackage[utf8]{inputenc}

\title{ME19B065}
\author{Amit Kumar me19b065}
\date{July 2021}

\begin{document}

\maketitle

\section{My Favourite Physics Equation}
\hspace{6mm}
\begin{equation}
\label{eqn:main_eq}
\Large
      \frac{-\hbar^2}{2m}\Delta^2{\Psi} + V\Psi = i\hbar{\frac{\partial \Psi}{\partial t}}
\end{equation}
\\
In simple words, the equation represents that the total energy of a particle is the sum of its kinetic and potential energy. \\
\begin{equation}
\label{eqn:sim_eq}
\Large
    KE+PE=TE
\end{equation}
\\
Each terms of equation \ref{eqn:main_eq} and \ref{eqn:sim_eq} can be compared easily. Hence, schrödinger's wave equation is nothing but an Energy Conservation Equation.


\end{document}
